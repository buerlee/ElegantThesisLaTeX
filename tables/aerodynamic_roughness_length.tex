% Table generated by Excel2LaTeX from sheet 'Sheet1'
\index{boundary conditions!roughness length}

\begin{table}[htbp]
\footnotesize
\centering
\captionsetup{format=plain}	
	\caption[Davenport classification of effective terrain roughness]{Davenport classification of effective terrain roughness (revision 2000) \textemdash $z_0$ is called the aerodynamic roughness length \citep{Wallace2006}.}
	\begin{tabular}{@{}lp{9cm}S@{}}
		\addlinespace
		\toprule
		Classification & Landscape & {$z_0 [m]$} \\
		\midrule
		Sea   & Calm sea, paved areas, snow-covered falt plain, tide flat, smooth desert. & 0.0002 \\
		Smooth & Beaches, pack ice, morass, snow-covered fields. & 0.005 \\
		Open  & Grass prairie or farm fields, tundra, airports, heather. & 0.03 \\
		Rougly open & Cultivated area with low crops and occasional obstacles (single bushes). & 0.1 \\
		Rough & High crops, crops of varied height, scattered obstacles such as trees or hedgerows, vineyards. & 0.25 \\
		Very rough & Mixed farm fields and forest clumps, orchards, scattered buildings. & 0.5 \\
		Closed & Regular coverage with large size obstacles with open spaces roughly equal to obstacle heights, suburban houses, villages, mature forests. & 1.0 \\
		Chaotic & Centers of large towns and cities, irregular forests with  scattered clearings. & 2 \\
		\bottomrule
	\end{tabular}%
	\label{tab:davenport_roughness_class}%
\end{table}%
